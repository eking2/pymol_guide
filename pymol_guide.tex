\documentclass[aspectratio=169]{beamer}
\usepackage[utf8]{inputenc}
\usepackage{listings,lstautogobble}
\usepackage{bookmark}
\usetheme{Pittsburgh}

\lstset{basicstyle=\ttfamily\footnotesize,
        moredelim=**[is][\color{blue}]{@}{@},
        autogobble=true,
        breaklines=true}

\beamertemplatenavigationsymbolsempty
\setbeamertemplate{footline}[frame number]
\setbeamertemplate{frametitle}[default][left]
%\setbeamersize{text margin left=0.5cm,text margin right=0cm} 

\title{Han Li Lab PyMol Guide}
\institute{University of California, Irvine}

\begin{document}

\begin{frame}[noframenumbering,plain]
    \maketitle
\end{frame}

\begin{frame}
    \frametitle{Table of Contents}
        \tableofcontents
\end{frame}

\section{Object Selection}

\begin{frame}[fragile]
\frametitle{Selection Shortcuts}
    \begin{lstlisting}
        sel <variable_name>, <object>          @# selection template@
        remove solv                            @# remove solvent@
        remove inorg                           @# remove ions@
        remove not chain <chain>               @# remove other chains@
        sel <ligand>, org                      @# select all ligands@
        sel <protein>, poly                    @# select all protein atoms@
        sel <target>, rep sticks               @# select by representation@
        sel <residue>, resi <residue number>   @# select residue by number@
        sel <residue>, resn <residue name>     @# select object by residue name@
        sel <atom>, name <atom name>           @# select atom by atom name@
        reset                                  @# reset camera origin@

        @# Examples@
        @# selects residues 23, 54, 55, 56, 57, 99@
        sel target_residues, resi 23+54-57+99    

        @# selects ligand with 3-letter code NAD from chain A@ 
        sel target_ligand, resn NAD and chain A

    \end{lstlisting}
\end{frame}


\begin{frame}[fragile]
\frametitle{Displaying Binding Pocket Interactions}
    \begin{columns}
        \begin{column}{0.50\textwidth}
            \includegraphics[scale=0.11]{images/ldh_pocket.png}
        \end{column}
        \begin{column}{0.45\textwidth}
           \begin{lstlisting}[mathescape=true]
                @# all residues with any atom around 3.5 angstrom@
                sel pocket, br. org around 3.5
                show sticks, pocket
                color cyan, pocket
                util.cnc
                dist hbonds, org, poly, mode=2
                hide labels
            \end{lstlisting}
        \end{column}
    \end{columns}
\end{frame}

\section{Structure Manipulation}

\begin{frame}[fragile]
\frametitle{Alignment}
    \begin{lstlisting}
    @# 1 to 1 alignment@
    fetch 1j49; fetch 4e5n
    remove solv; remove not chain A; remove inorg
    align 1j49, 4e5n  @# fits 1j49 onto 4e5n@
    reset

    @# multiple to 1 alignment@
    fetch 1j49; fetch 4e5n; fetch 6ih4
    remove solv; remove not chain A; remove inorg
    alignto 4e5n  @# align all to 4e5n@
    reset
    \end{lstlisting}

\end{frame}

\begin{frame}[fragile]
\frametitle{Ligand Transfer}
    \begin{lstlisting}
    fetch 1qi1; fetch 1qi6         @# 1qi1 gapn with nadp, 1qi6 apo gapn@
    remove not chain A
    remove solv
    remove inorg
    align 1qi1, 1qi6               @# align binding pockets@
    extract nadp, resn NAP
    sel 1qi6_nadp, 1qi6 + nadp
    save 1qi6_nadp.pdb, 1qi6_nadp  @# or save with File -> Export Molecule@
    @# type pwd to see the default directory the file is saved to@
    \end{lstlisting}

\end{frame}

\begin{frame}[fragile]
\frametitle{Mutagenesis}

Follow instructions from the PyMol website
\href{https://pymolwiki.org/index.php/Mutagenesis}{\color{blue}{https://pymolwiki.org/index.php/Mutagenesis}}

\vspace*{0.5cm}
\begin{enumerate}
    \item{Wizard $\rightarrow$ Mutagenesis $\rightarrow$ Protein}
    \item{Select target residue by clicking}
    \item{Select mutation residue in Mutagenesis box}
    \item{Examine rotamers with arrows on bottom right}
    \item{Apply to save}
\end{enumerate}

\end{frame}

\section{Publication Quality Images}
\begin{frame}[fragile]
\frametitle{General Settings}
    \begin{lstlisting}
        bg white                       @# change background color@
        set ray_opaque_background, 1   @# make background solid@
        set ray_shadow, 0              @# turn off shadows@
        set antialias, 2               @# smooth edges@
        set cartoon_fancy_helices, 1   @# adds ridges to helix edges@
        util.cnc                       @# colors oxygen and nitrogen atoms@
        ray <width>                    @# ray and scale resolution@

        @# Recommended settings@
        set ray_trace_mode, 3          @# optional neon colors and outline@
        color white, poly
        color green, org
        color cyan, poly and rep sticks

    \end{lstlisting}
\end{frame}

\begin{frame}[fragile]
\frametitle{Showing Backbone Atoms}
    \begin{columns}
        \begin{column}{0.50\textwidth}
            \includegraphics[scale=0.10]{images/side_chain_helper.png}
        \end{column}
        \begin{column}{0.50\textwidth}
            \includegraphics[scale=0.10]{images/side_chain_helper_off.png}
        \end{column}
    \end{columns}
    \begin{lstlisting}[xleftmargin=0.2\textwidth]
    fetch 1j49
    [...]  @# clean up@
    dist hbonds, org, resi 234, mode=2
    hide labels
    set cartoon_side_chain_helper, 0, resi 234
    \end{lstlisting}

\end{frame}


\begin{frame}[fragile]
\frametitle{Glucose Dehydrogenase}
    \begin{columns}
        \begin{column}{0.50\textwidth}
            \includegraphics[scale=0.12]{images/gludh_copy.png}
        \end{column}
        \begin{column}{0.50\textwidth}
            \begin{lstlisting}
                load gludh_80_nmn.pdb
                bg white
                set ray_opaque_background, 1
                set cartoon_fancy_helices, 1
                set ray_shadow, 0
                color white, poly
                color orange, org
                show cart
                show surf
                set transparency, 0.7
                sel mutations, i. 93+39+195+17 and name ca
                show spheres, mutations
                color deeppurple, rep spheres
                set sphere_scale, 0.5
                util.cnc org
            \end{lstlisting}
        \end{column}
    \end{columns}
\end{frame}

\begin{frame}[fragile]
\frametitle{Lactate Dehydrogenase}
    \begin{columns}
        \begin{column}{0.50\textwidth}
            \includegraphics[scale=0.10]{images/1j49_ade.png}
        \end{column}
        \begin{column}{0.50\textwidth}
            \vspace*{-0.5cm}
            \begin{lstlisting}
                fetch 1j49 
                bg white; color white, poly
                remove chain A; remove inorg; remove solv
                set ray_opaque_background, 1
                set ray_trace_mode, 3
                set cartoon_transparency, 0.3
                color green, org
                show lines, br. org around 3.5
                color silver, rep lines
                sel spec_loop, resi 176-178
                show sticks, spec_loop
                color cyan, resi 176
                color violet, resi 177
                color lightorange, resi 178
                dist hbonds, resi 176, all, mode=2
                hide labels; show lines, resi 154
                set cartoon_side_chain_helper, 0, resi 154+177+178
                util.cnc
            \end{lstlisting}
        \end{column}
    \end{columns}
\end{frame}

\begin{frame}[fragile]
\frametitle{Nitrogenase Iron Protein}
    \begin{columns}
        \begin{column}{0.50\textwidth}
            \includegraphics[scale=0.105]{images/6nzj_3.png}
        \end{column}
        \begin{column}{0.50\textwidth}
            \begin{lstlisting}
                fetch 6nzj
                bg white 
                color lightblue, chain A
                color palegreen, chain B
                util.cnc
                remove resn SO4; remove solv
                set ray_opaque_background, 1
                @# outline with regular colors@
                set ray_trace_mode, 1  
                set cartoon_transparency, 0.7
                show sticks, inorg
                set sphere_scale, 0.4
                set cartoon_transparency, 0.5
                show sticks, br. inorg around 3.5
                hide sticks, resi 96
            \end{lstlisting}
        \end{column}
    \end{columns}
\end{frame}

\section{Rotation Movies}
\begin{frame}[fragile]
    \frametitle{Rotation Movies}

    Common options under Movie $\rightarrow$ Program 

    \vspace*{1.5cm}
    GluDH

    \begin{lstlisting}
    movie.add_roll(16.0, axis=`y', start=1)
    File -> Export Movie
    \end{lstlisting}

    \href{https://youtu.be/eaRJjZ0OLRs}{\color{blue}{https://youtu.be/eaRJjZ0OLRs}}
\end{frame}

\end{document}
